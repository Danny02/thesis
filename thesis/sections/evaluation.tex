\section{Evaluation}
Code formatting is inherently a subjective topic.
This introduces a challenge when evaluating a code formatter.
In this chapter, we will present measurements that we believe show the success of scalafmt.
We do not measure how well software developers perceive scalafmt formatted code.
Instead, we will focus on \emph{performance benchmarks} and \emph{user adoption}.

\subsection{Performance benchmarks}
In this chapter, we describe our test methodology, key metrics and the performance results.

\subsubsection{Test methodology}
We run micro benchmarks to get insight into how scalafmt performs in an interactive software developer workflow where scalafmt runs on file save.
We select four files of varying sizes: small ($\sim$ 50 LOC), medium ($\sim$300 LOC), large ($\sim$1.000 LOC) and extra large ($\sim$4.500 LOC).
We use the OpenJDK Java Microbenchmark Harness (JMH)\autocite{OpenJ38:online} to measure the time it takes format the files.
JMH can be used to build, run and analyze benchmarks written in languages targeting the JVM, including Scala.
The sbt-jmh\autocite{ktoso84:online} plugin makes it easy to integrate JMH with a Scala project.
For accurate measurements, we run 10 warmup iterations followed by 10 measured iterations.
We compare the running time with Scalariform, an alternative code formatter for Scala discussed in section~\ref{sec:scalariform}.
The micro benchmark is single threaded.

We run macro benchmarks to get insights into how scalafmt performs in a continuous integration setup.
We format 9.423 source files taken from a subset of seven projects from the property based testing inputs\footnote{
  Download here \url{https://github.com/olafurpg/scalafmt/releases/download/v0.1.4/repos.tar.gz}
}: intellij-scala, akka, spark, scalding, fastparse, scala-js and goose.
The whole sample contains a total of 1.219.235 lines of code.
Table~\ref{tab:macro_sample} shows the distribution of file sizes.
\begin{table}
  \centering
  \csvautobooklongtable{data/file_sizes.csv}
  \caption{Lines of code per file in macro benchmark}~\label{tab:macro_sample}
\end{table}
Observe that over 90 percent of all files contain under 300 lines of code.
Less than one percent of all files contain over 1.000 lines of code.
For accurate measurements, we run three iterations of the macro benchmark.
As with the micro benchmark, we compare the running time with Scalariform.
The macro benchmark is multi-threaded and takes advantage of all cores on the underlying hardware.

Our hardware is a Macbook Pro (Retina, 15-inch, Mid 2014) laptop with a quad-core 2.5 GHz Intel Core i7 processor, 256 KB L2 cache per core and 6 MB shared L3 cache.
The laptop has 16 GB 1600 MHz DDR3 memory.
The operating system is OS X El Capitan 10.11.5.
We run the benchmarks from the scalafmt commit id \href{https://github.com/olafurpg/scalafmt/tree/aff5e794dae4787b08243f8abb87a3ca4d907e40}{aff5e794} compiled against Scala 2.11.7, running on JVM version 8, update 91.

\subsubsection{Results}
Table~\ref{tab:micro} shows the results from the micro benchmark.
\begin{table}\label{tab:micro}
  \centering
  \caption{Results from micro benchmark}
  \csvautobooklongtable{data/file_sizes.csv}
\end{table}

Table~\ref{tab:macro} shows the results from the macro benchmark.
\begin{table}[H]
  \centering
  \begin{tabular}{lllll}
  Benchmark                &  Cores  &   Score  &   Error &  Units\\
  \hline
  \hline
  Parallel.scalafmt        &  4      &  14.616  &   0.632 &   s/op\\
  Parallel.scalariform     &  4      &   2.810  &   0.641 &   s/op\\
\hline
  Ratio &  & 5.20 &  & \\
  \\
  Synchronous.scalafmt     &  1      &  35.654  &   0.459 &   s/op\\
  Synchronous.scalariform  &  1      &   5.951  &   0.135 &   s/op\\
\hline
  Ratio &  & 5.99 &  & 
\end{tabular}

  \caption{Results from macro benchmark}\label{tab:macro}
\end{table}
Observe that the total time difference lower than all other columns.
This is the result of the parallel execution of the benchmark, which means that the scalafmt test run utilized more CPU wall time than the Scalariform benchmark.



\subsection{Adoption}\label{sec:adoption}
TODO


