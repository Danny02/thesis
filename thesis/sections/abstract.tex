
\begin{abstract}
  Code formatters bring many benefits to software development such as enforcing a consistent coding style across teams, more effective code reviews and enabling automated large-scale refactoring.
  % Still, code formatters can be tricky to get right.
  This thesis addresses how to develop a code formatter for the Scala programming language.
  We present scalafmt, an opinionated Scala code formatter that captures many popular Scala idioms and coding styles.
  This thesis introduces language-agnostic algorithms and tooling that scalafmt uses to implement advanced features such as line wrapping and configurable vertical alignment.
  We have validated that these techniques work well in practice.
  Scalafmt has been installed over 6.500 times in only 3 months and
  several popular open-source libraries have chosen to reformat their codebases with scalafmt.
\end{abstract}
% \newpage
\vspace{1in}
\renewcommand{\abstractname}{Útdráttur}

\begin{abstract}
  Kóðasnyrtar (e. code formatters) eru nytsamleg tól í hugbúnaðarþróun.
  Helstu kostir kóðasnyrta eru meðal annars að geta sjálfvirkt framfylgt samræmdum forritunarstíl, skilvirkari kóðaumsögnum og gera kleift að endurskipleggja stór forritasöfn.
  Þetta verkefni fjallar um að þróa kóðasnyrti fyrir Scala forritunarmálið.
  Við kynnum scalafmt, kóðasnyrti fyrir Scala sem fangar marga vinsæla Scala forritunarstíla og forritunartiltæki.
  Þetta verkefni kynnir reiknirit og gagnagrindur til að útfæra háþróaða eiginlega eins og að brjóta langar forritunarskipanir á einni línu niður í margar línur og raða tóka af svipuðu tagi frá mörgum línum þannig að tókarnir liggi á sama lóðrétta dálki.
  Aðferðir sem kynntar eru í þessu verkefni hafa sannað sig í verki.
  Scalafmt hefur verið halað niður yfir 6.500 sinnum á eingöngu 3 mánuðum og fjöldi af vinsælum opnum forritnsöfnum hafa kosið að snyrta kóðann sinn með scalafmt.
\end{abstract}
