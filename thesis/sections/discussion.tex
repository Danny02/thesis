
\section{Future work}
Scalafmt is not free from issues.
The main areas for future work regard how to more soundly produce an optimal formatting layout and how to obtain better performance in an interactive developer workflow.

We experienced a long tail of problems while getting the best-first search to reach the last token for a lot of Scala code that we found in the wild.
If the search cannot reach the last token, scalafmt cannot format the file.
Although the concept of policies does give a lot of flexibility to concisely express different formatting layouts, our experience is that it can be easy to create overly strict policies that eliminate all active search states.
It is worth to explore more principal approaches on how to define formatting layouts so that we can better guarantee that the search completes successfully every time.
We believe \rfmt{}s approach of combining the convenience of combinators with algebraic properties and optimized dynamic programming for excellent performance opens an interesting venue to solve this problem.

Incremental formatting opens an opportunity to get ludicrous formatting speed in an interactive developer workflow.
Incremental formatting works like incremental compilation in the sense that instead of formatting a whole file on every invocation, the formatter reuses output from previous invocations and only reformat the lines that have changed.
We believe that incremental formatting could cut down the formatting time for a large source file by several orders of magnitude, for example from 2s to 20ms.

\section{Conclusion}
We set out to implement a code formatter for the Scala programming language that supports several important features: an opinionated setting, a maximum line length setting, vertical alignment and fast performance.
In this thesis, we have presented data structures and algorithms that we used to develop scalafmt, a Scala code formatter that supports the first three required features and goes far towards achieving good performance.
Benchmarks reveal that scalafmt is 6x slower than Scalariform, an alternative Scala code formatter.
However, the quick user adoption of scalafmt indicates that scalafmt's current performance is already acceptable for many software developers.
We have reason to believe there is still plenty of room to improve on scalafmt's performance in order to meet the needs of the most performance demanding users.
