\section{Introduction} % (fold)
\label{sec:Introduction}
\lstset{style=scala}
Without code formatters, software developers are responsible for manipulating all syntactic trivia in their programs.
What is syntactic trivia?
Consider the Scala code snippets in listings~\ref{lst:unformatted} and~\ref{lst:formatted1}.

\begin{minipage}{.45\textwidth}
\lstinputlisting[label={lst:unformatted}, caption=Unformatted code]{code/unformatted.scala}
\end{minipage}
\hfil
\begin{minipage}{.45\textwidth}
\lstinputlisting[label={lst:formatted1}, caption=Formatted code]{target/formatted1.scala}
\end{minipage}

Both snippets represent the same application.
The only difference lies in where the programmer has chosen to break lines.
Characters such as spaces and line breaks that do not affect the execution of the program are syntactic trivia.
Although syntactic trivia has no meaning for the execution of the program, listing~\ref{lst:formatted1} is arguably easier to understand, maintain and extend for the software developer.
A code formatter is a tool that automatically converts a program such as in listing~\ref{lst:unformatted} into a readable and maintainable program such as in listing~\ref{lst:formatted1}.
Code formatting brings several benefits to software development.

Code formatting enables automated large-scale refactoring.
Google used ClangFormat\autocite{jasper_clangformat_2013}, a C++ code formatter,
in the process of migrating a large legacy C++98 codebase to use the modern C++11 standard\autocite{wright_large-scale_2013}.
The automatically refactored code was formatted with ClangFormat to ensure that it adhered to Google's strict coding style\autocite{_google_????}.
Similar migrations can be expected in the near future for the Scala community once new dialects, such as Dotty\autocite{rompf_f_2015}, gain popularity.

Code formatting is valuable in collaborative coding environments.
The Scala.js project\autocite{_scala.js_????} has over 40 contributors and the Scala.js coding style\autocite{doeraene_scala.js_2015} --- which each Scala.js contributor is expected to know by heart --- is defined at a whopping 2.600 word count.
Each contributed patch is manually verified against the coding style by the project maintainers.
This adds a burden on both contributors and maintainers.
Several prominent Scala community members have raised this issue.
ENSIME\autocite{_ensime_????} is a popular Scala interaction mode for text editors such as Vim and Emacs.
Sam Halliday, an ENSIME maintainer, says ``I don't have time to talk about formatting in code reviews. I want the machine to do it so I can focus on the design.''\autocite{halliday_i_2016-1}.
Akka\autocite{_akka_????} is a popular concurrent and distributed programming library for Scala with over 300 contributors.
Viktor Klang, a maintainer of Akka, suggests a better alternative: ``Code style should not be enforced by review, but by automate rewriting. Evolve the style using PRs against the rewriting config.''.\autocite{klang_code_2016}.

With scalafmt, we hope to relieve Scala developers from the burden of manipulating syntactic trivia so they can instead direct their full attention to writing correct, maintainable and fast code.

\subsection{Research objective}
% Code formatters bring several benefits to software development.
% However, the Scala community has enjoyed limited benefits from the 
% Can we develop a code formatter for the Scala programming language

What algorithms and data structures allow us to develop a code formatter for the Scala programming language with a \emph{maximum line length setting}, \emph{opinionated setting}, \emph{vertical alignment} and \emph{good performance}?
Those features are defined as follows:
\begin{itemize}
  \item \textbf{Maximum line length setting}: a code formatter with a maximum line length setting ensures that each line in the formatted output contains no more than a certain number of characters.
    Many coding styles enforce a maximum line length to ensure code is readable from different environments such as small split screens and code review interfaces.
  \item \textbf{Opinionated}:
    An opinionated setting is a prerequisite to enforce a uniform coding style.
    An opinionated code formatter takes liberty to disregard line breaks and other formatting decisions in the original source input to ensure that formatted source files follow the same line breaking conventions.
  \item \textbf{Vertical alignment}: vertical alignment is a formatting convention where redundant whitespace is added before a token to align it on the same vertical column as similar tokens from other lines. Many Scala coding styles enforce vertical alignment to enhance code readability.
  \item \textbf{Performance}: Users expect code formatters to run fast.
    In the most demanding settings, a code formatter needs to be able to format thousands of lines of code in at most a few hundred milliseconds.

\end{itemize}

\subsection{Contributions}
The main contributions presented in this thesis are the following:
\begin{itemize}
  \item language-agnostic algorithms and data structures to implement line
    wrapping with a maximum line length setting --- in our opinion, the most
    challenging part of developing a code formatter --- as well as configurable
    vertical alignment.
    This work is presented in section~\ref{sec:algorithms}.
  \item methods to optimize and test the developed algorithms.
    This work is presented in section~\ref{sec:tooling}.
  \item practical validation of developed algorithms and methods with the scalafmt code formatter.
    The empirical results of this validation are presented in section~\ref{sec:evaluation}.
\end{itemize}
